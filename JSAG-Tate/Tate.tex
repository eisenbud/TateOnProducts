\documentclass[twoside,12pt, leqno]{amsart}
\usepackage{amsmath,amscd,amsthm,amssymb,amsxtra,latexsym,epsfig,epic,graphics}
\usepackage[matrix,arrow,curve]{xy}
\usepackage{graphicx}
\usepackage{diagrams}
%\usepackage{pgfplots}
\usepackage{tikz}  %TikZ
\usepackage{color} 
\usetikzlibrary{arrows,calc} 
\usetikzlibrary{decorations.pathmorphing}
\usetikzlibrary{decorations.markings} 
\usetikzlibrary{decorations.pathreplacing} 
\usetikzlibrary{plothandlers}


%\usepackage{amsrefs}
%%%%%%%%%%%%%%%%%%%%%%%%%%%%%%%%%%%%%%%%%
%\textwidth16cm
%\textheig\codim20cm
%\topmargin-2cm
\oddsidemargin.8cm
\evensidemargin1cm

%%%%%Definitions
\input preamble.tex
\def\e{{\epsilon}}
\def\TU{{\bf U}}
\def\AA{{\mathbb A}}
\def\BB{{\mathbb B}}
\def\bB{{\mathbb B}}
\def\PP{{\mathbb P}}
\def\P{{\mathbb P}}
\def\QQ{{\mathbb Q}}
\def\FF{{\mathbb F}}
\def\facet{{\bf facet}}
\def\image{{\rm image}}
\def\name{{\rm name}}
\def\cE{{\cal E}}
\def\cF{{\cal F}}
\def\cG{{\cal G}}
\def\cH{{\cal H}}
\def\cHom{{{\cal H}om}}
\def\fix#1{{\bf ***Fix:} #1 {\bf ***}}
\def\david#1{{\bf *** David:} #1 {\bf ***}}
\DeclareMathOperator{\rH}{{\rm H}}
\def\fC{{\mathfrak C}}
\def\Tr{{\rm Tr}}
\def\bC{{\mathbb C}}
\def\Gr{{\rm Gr}}
\def\CI{{\mathcal I}}
\def\CH{{\mathcal H}}
%\def\CCH{{\mathcal {CNT}}}
\def\CCH{{\mathcal {HC}}}
\def\rH{{\rm H}}

\def\soc{{\rm soc\,}}
\def\jacobian{{\rm Jac}}
\def\Rbar{{\overline R}}
\def\Ibar{{\overline I}}
\def\mm{{\frak m}}
\def\RR{{\mathcal R}}
\def\Trace{{\rm Tr}}

\def\CO{{\mathcal O}}
\def\CT{{\mathcal T}}
\def\CHom{{\mathcal Hom}}
\def\Spec{{{\rm Spec}\,}}
\def\cone{{{\rm cone}\,}}

\def\tR{{\tilde R}}
\def\tI{{\tilde I}}
\def\tJ{{\tilde J}}
\def\tK{{\tilde K}}
\def\tH{{\tilde H}}
\def\tF{{\tilde F}}

\newarrow{Iso} -----

\def\Abar{{\overline A}}
\def\Rbar{{\overline R}}
\def\Ibar{{\overline I}}
\def\Jbar{{\overline J}}
\def\Kbar{{\overline K}}
\def\abar{{\overline \alpha}}
\def\bbar{{\overline \beta}}
\def\m{{\frak m}}
\def\Rbar{{\overline R}}

\def\gr{{\rm gr}}
\def\init{{\rm in}}


\def\frank#1{{\bf *** Frank:} #1 {\bf ***}}
\def\david#1{{\bf *** David:} #1 {\bf ***}}
\def\lbracket{{[\kern-1.5pt[}}
\def\rbracket{{]\kern-1.5pt]}}

\def\seq#1#2{{#1_{1},\dots,#1_{#2}}}
\def\ff#1{{f_{1},\dots, f_{#1}}}

\makeatletter
\def\Ddots{\mathinner{\mkern1mu\raise\p@
\vbox{\kern7\p@\hbox{.}}\mkern2mu
\raise4\p@\hbox{.}\mkern2mu\raise7\p@\hbox{.}\mkern1mu}}
\makeatother


%%%%%%%%%%%%%%%%%%Silvio's macros for the diagrams
\usepackage{times}
\newdimen\x \x=12pt

%\usepackage{mat\codimime}
\usepackage{color}

%\usepackage{color}
%\usepackage[usenames,dvipsnames,svgnames,table]{xcolor}

\usepackage[breaklinks,bookmarksopen,bookmarksnumbered,urlcolor=blue]{hyperref}
\hypersetup{colorlinks=true,backref=true,citecolor=blue}

%\pagestyle{MYheadings}
%\date{April 2013-December 2015}
\author[David Eisenbud]{David Eisenbud}
\address{Department of Mathematics, University of California at Berkeley and the Mathematical
Sciences Research Institute, Berkeley, CA 94720, USA}
\email{de@msri.org}

\author{Daniel Erman}
\address{Department of Mathematics, University of Wisconsin,
	Madison, WI, ****, USA}
\email{derman@math.wisc.edu}


\author[Frank-Olaf Schreyer]{Frank-Olaf Schreyer}
\address{Fachbereich Mathematik, Universit\"at des Saarlandes, Campus E2 4, D-66123 Saar\-br\"ucken, Germany}
\email{schreyer@math.uni-sb.de}



\title{Tate Resolutions on Products of Projective Spaces: Cohomology and PushForward Complexes}
\begin{document}

\begin{abstract}
We describe the  Macaulay2 package TateOnProducts and its capabilities, which include computing cohomology tables of sheaves
on products of projective spaces and the derived category pushForward of a sheaf under a morphism from a projective scheme to
a projective space.
\end{abstract}

\maketitle

\section*{Introduction} 
\let\thefootnote\relax\footnote{
\noindent AMS Subject Classification:
Primary: 14H99 ????,
Secondary: 13D02????, 14H51???? \smallbreak
Keywords: *******\smallbreak
The first author is grateful to the
National Science Foundation for partial support. This work is a contribution to Project I.6 of the second author within the SFB-TRR 195 "Symbolic Tools in Mathematics and their Application" of the German Research Foundation (DFG).}


\section*{Introduction}

\section{The Tate Resolution and its properties}

strands etc

\section{Computing Cohomology Tables}

illustrate the functions

\section{ $R\pi_*$ of a Sheaf}

illustrate the function

\bibliographystyle{ABC99}
\begin{thebibliography}{ABC99}

%\bibitem{Bass1963} Bass, Hyman
%On the ubiquity of Gorenstein rings.
%Math. Z. 82 1963 8\UTF{2013}28.



%\bibitem[A05]{A} M.~Aprodu: Remarks on syzygies of d-gonal curves Green's conjecture for curves,  Math. Res. Lett. 12 (2005), no. 2-3, 387--400.
%
%\bibitem[AF11]{AF} M.~Aprodu and G.~Farkas: Green's conjecture for curves on arbitrary {$K3$} surfaces, Compos. Math. 147 (2011), 839--851.
%
% \bibitem[AFPRW]{AFPRW} M.~Aprodu, G.~Farkas, S.~Papadima, C.~Raicu and J.~Weyman: In preparation.
%
%\bibitem[AB58]{AB} M.~Auslander and D.~Buchsbaum: Codimension and Multiplicity. Annals of Mathematics 68 (1958) 625--657.
%
%
%\bibitem[BE95]{BE} D.~Bayer and D.~Eisenbud: Ribbons and their canonical embeddings. Trans. Amer. Math. Soc. 347 (1995) 719--756. 
%
%
%\bibitem[BS15]{BS15}C.~Berkesch and F.-O.~Schreyer:
%Syzygies, finite length modules, and random curves, in:
%Commutative algebra and noncommutative algebraic geometry, {V}ol. {I},
%Math. Sci. Res. Inst. Publ.,
%67 (2015),
%25--52.
%
%\bibitem[B17]{Bopp} C.~Bopp: Canonical curves, Scrolls and K3 surfaces. \href{https://publikationen.sulb.uni-saarland.de/bitstream/20.500.11880/26917/1/phd_bopp.pdf}{Dissertation, Saarbr\"ucken Fall 2017}.
%
%\bibitem[BH15]{BH15} C.~Bopp and M.~Hoff: Resolutions of general canonical curves on rational normal scrolls. Archiv der Mathematik (Basel) 105 (2015) 239--249.
%
%\bibitem[BH17]{BH17} C.~Bopp and M.~Hoff: Moduli of lattice polarized K3 surfaces via relative canonical resolutions. Preprint \href{https://arxiv.org/abs/1704.02753}{arXiv:1704.02753}.
%
%
%
%
%\bibitem[BS18]{BS18} C.~Bopp and F.-O.~Schreyer:  A version of Green's conjecture over fields of finite characteristic. Preprint \href{https://arxiv.org/abs/1803.10481}{arXiv:1803.10481}.
%
%\bibitem[BH93]{BH93} W.~Bruns and J.~Herzog, J\"urgen:
%Cohen-{M}acaulay rings
%Cambridge Studies in Advanced Mathematics 39, Cambridge University Press. xi, 403 p. (1993). 
%
%\bibitem[D15]{D} A.~Deopurkar: The canonical syzygy conjecture for ribbons. \href{https://arxiv.org/abs/1510.07755}{arXiv:1510.07755}
%
%\bibitem[DFS16]{DFS} A.~Deopurkar, M.~ Fedorchuk and D.~Swinarski: Toward GIT stability of syzygies of canonical curves.
% Algebr. Geom.  3  (2016),  no. 1, 1--22.
%
%\bibitem[E97]{E} D.~Eisenbud: Commutative Algebra with a View toward Algebraic Geometry. GTM 150, Springer-Verlag NY, 1997.
%
%\bibitem[E05]{E05} D.~Eisenbud: The Geometry of Syzygies. GTM 229, Springer-Verlag NY, 2005.
%
%\bibitem[EH87]{EH} D.~Eisenbud and J.~Harris: Varieties of Minimal Degree (a centennial account). Algebraic geometry, 
%Proc. Sympos. Pure Math., 46, Part 1, pp. 3-13. Amer. Math. Soc., Providence, RI, 1987. 
%
%
%
%\bibitem[EiSa]{EiSa} D.~Eisenbud and A.~Sammartano: Correspondence Scrolls. In preparation.
%
%\bibitem[ES18]{ES18} D.~Eisenbud and F.-O.~Schreyer: K3Carpets , a Macaulay2 package to investigate K3 carpets.
%Available at\\ \href{https://www.math.uni-sb.de/ag/schreyer/index.php/computeralgebra}{https://www.math.uni-sb.de/ag/schreyer/index.php/computeralgebra}.
%
%\bibitem[EMSS16]{EMSS} B.~Erocal, O.~Motsak, F.-O.~Schreyer and A.~Steenpass: 
%Refined Algorithms to Compute Syzygies. J. Symb. Comput. 74 (2016), 308--327.
%
%\bibitem[F69]{F69} W.~Fulton: Hurwitz schemes and irreducibility of moduli of algebraic curves. Ann. Math. (2)
%90 (1969), 542--575.
%
%\bibitem[GP97]{GP}  F.J.~Gallego and B.~Purnaprajna, Degenerations of K3 surfaces in projective space.
%Trans. Amer. Math. Soc. 349 (1997) 2477--2492.
%
%\bibitem[M2]{M2} D.R.~ Grayson and M.E.~ Stillman,
%          Macaulay2, a software system for research in algebraic geometry.
%          Available at \href{https://faculty.math.illinois.edu/Macaulay2/}{https://faculty.math.illinois.edu/Macaulay2/}
%      
%\bibitem[M95]{Muk}   S.~Mukai, Curves and symmetric spaces I. Amer. J. Math 117 (1995), 1627--1644.
%
%\bibitem[S86]{S86} F.-O.~Schreyer: Syzygies of canonical curves and special linear series. Math. Ann. 275 (1986), 105--137.
%
%\bibitem[S88]{S88} F.-O.~Schreyer: Green's conjecture for the general p-gonal curve of large genus. In: Algebraic curves and projective geometry, Proceedings Trento 1988, Springer Lecture Notes 1389, 254--260.
%
%\bibitem[S91]{S91} F.-O.~Schreyer: A standard basis approach to syzygies of canonical curves. J. reine angew. Math. 421 (1991), 83--123.
%
%
%\bibitem[V05]{V05} C.~Voisin: 
%Green's canonical syzygy conjecture for generic curves of odd genus. Compos. Math. 141 (2005) 1163--1190. 

\end{thebibliography}

%\bigskip
%
%%\author[David Eisenbud]{David Eisenbud}
%%\address{ Department of mathematics, University of California at Berkeley, Berkeley, CA 94720, USA, and 
%%The Mathematical Sciences Research Institute}
%%\email{de@msri.org}
%%
%%\author[Frank-Olaf Schreyer]{Frank-Olaf Schreyer}
%%\address{Fachbereich Mathematik, Universit\"at des Saarlandes, Campus E2 4, D-66123 Saar\-br\"ucken, Germany}
%%\email{schreyer@math.uni-sb.de}
%%
%%\vbox{\noindent Author Addresses:\par
%%\smallskip
%%\noindent{David Eisenbud}\par
%%\noindent{Mathematical Sciences Research Institute,
%%Berkeley, CA 94720, USA}\par
%%\noindent{de@msri.org}\par
%%}

\end{document}



