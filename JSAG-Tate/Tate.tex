\documentclass[twoside,12pt, leqno]{amsart}
\usepackage{amsmath,amscd,amsthm,amssymb,amsxtra,latexsym,epsfig,epic,graphics}
\usepackage[matrix,arrow,curve]{xy}
\usepackage{graphicx}
\usepackage{diagrams}
%\usepackage{pgfplots}
\usepackage{tikz}  %TikZ
\usepackage{color} 
\usetikzlibrary{arrows,calc} 
\usetikzlibrary{decorations.pathmorphing}
\usetikzlibrary{decorations.markings} 
\usetikzlibrary{decorations.pathreplacing} 
\usetikzlibrary{plothandlers}


%\usepackage{amsrefs}
%%%%%%%%%%%%%%%%%%%%%%%%%%%%%%%%%%%%%%%%%
%\textwidth16cm
%\textheig\codim20cm
%\topmargin-2cm
\oddsidemargin.8cm
\evensidemargin1cm

%%%%%Definitions
\input preamble.tex
\def\e{{\epsilon}}
\def\TU{{\bf U}}
\def\AA{{\mathbb A}}
\def\BB{{\mathbb B}}
\def\bB{{\mathbb B}}
\def\PP{{\mathbb P}}
\def\P{{\mathbb P}}
\def\QQ{{\mathbb Q}}
\def\FF{{\mathbb F}}
\def\facet{{\bf facet}}
\def\image{{\rm image}}
\def\name{{\rm name}}
\def\cE{{\cal E}}
\def\cF{{\cal F}}
\def\cG{{\cal G}}
\def\cH{{\cal H}}
\def\cHom{{{\cal H}om}}
\def\fix#1{{\bf ***Fix:} #1 {\bf ***}}
\def\david#1{{\bf *** David:} #1 {\bf ***}}
\DeclareMathOperator{\rH}{{\rm H}}
\def\fC{{\mathfrak C}}
\def\Tr{{\rm Tr}}
\def\bT{{\bf T}}
\def\bU{{\bf U}}
\def\bC{{\mathbb C}}
\def\Gr{{\rm Gr}}
\def\CI{{\mathcal I}}
\def\CH{{\mathcal H}}
%\def\CCH{{\mathcal {CNT}}}
\def\CCH{{\mathcal {HC}}}
\def\rH{{\rm H}}
\def\soc{{\rm soc\,}}
\def\jacobian{{\rm Jac}}
\def\Rbar{{\overline R}}
\def\Ibar{{\overline I}}
\def\mm{{\frak m}}
\def\RR{{\mathcal R}}
\def\Trace{{\rm Tr}}

\def\CO{{\mathcal O}}
\def\CT{{\mathcal T}}
\def\CHom{{\mathcal Hom}}
\def\Spec{{{\rm Spec}\,}}
\def\cone{{{\rm cone}\,}}

\def\tR{{\tilde R}}
\def\tI{{\tilde I}}
\def\tJ{{\tilde J}}
\def\tK{{\tilde K}}
\def\tH{{\tilde H}}
\def\tF{{\tilde F}}

\newarrow{Iso} -----

\def\Abar{{\overline A}}
\def\Rbar{{\overline R}}
\def\Ibar{{\overline I}}
\def\Jbar{{\overline J}}
\def\Kbar{{\overline K}}
\def\abar{{\overline \alpha}}
\def\bbar{{\overline \beta}}
\def\m{{\frak m}}
\def\Rbar{{\overline R}}

\def\gr{{\rm gr}}
\def\init{{\rm in}}


\def\frank#1{{\bf *** Frank:} #1 {\bf ***}}
\def\david#1{{\bf *** David:} #1 {\bf ***}}
\def\lbracket{{[\kern-1.5pt[}}
\def\rbracket{{]\kern-1.5pt]}}

\def\seq#1#2{{#1_{1},\dots,#1_{#2}}}
\def\ff#1{{f_{1},\dots, f_{#1}}}

\makeatletter
\def\Ddots{\mathinner{\mkern1mu\raise\p@
\vbox{\kern7\p@\hbox{.}}\mkern2mu
\raise4\p@\hbox{.}\mkern2mu\raise7\p@\hbox{.}\mkern1mu}}
\makeatother


%%%%%%%%%%%%%%%%%%Silvio's macros for the diagrams
\usepackage{times}
\newdimen\x \x=12pt

%\usepackage{mat\codimime}
\usepackage{color}

%\usepackage{color}
%\usepackage[usenames,dvipsnames,svgnames,table]{xcolor}

\usepackage[breaklinks,bookmarksopen,bookmarksnumbered,urlcolor=blue]{hyperref}
\hypersetup{colorlinks=true,backref=true,citecolor=blue}

%\pagestyle{MYheadings}
%\date{April 2013-December 2015}
\author[David Eisenbud]{David Eisenbud}
\address{Department of Mathematics, University of California at Berkeley and the Mathematical
Sciences Research Institute, Berkeley, CA 94720, USA}
\email{de@msri.org}

\author{Daniel Erman}
\address{Department of Mathematics, University of Wisconsin,
	Madison, WI, ****, USA}
\email{derman@math.wisc.edu}


\author[Frank-Olaf Schreyer]{Frank-Olaf Schreyer}
\address{Fachbereich Mathematik, Universit\"at des Saarlandes, Campus E2 4, D-66123 Saar\-br\"ucken, Germany}
\email{schreyer@math.uni-sb.de}



\title{Tate Resolutions on Products of Projective Spaces: \\ Cohomology and PushForward Complexes}
\begin{document}

\begin{abstract}
We describe the  Macaulay2 package TateOnProducts and its capabilities, which include computing cohomology tables of sheaves
on products of projective spaces and the derived category pushForward of a sheaf under a morphism from a projective scheme to
a projective space.
\end{abstract}

\maketitle

\section*{Introduction} 
\let\thefootnote\relax\footnote{
\noindent AMS Subject Classification:
Primary: 14H99 ????,
Secondary: 13D02????, 14H51???? \smallbreak
Keywords: *******\smallbreak
The first author is grateful to the
National Science Foundation for partial support. This work is a contribution to Project I.6 of the second author within the SFB-TRR 195 "Symbolic Tools in Mathematics and their Application" of the German Research Foundation (DFG).}


\section*{Introduction}

This package exploits the Koszul duality between polynomial rings and exterior algebras to turn questions of cohomology of sheaves on a product of projective space into questions about free resolutions of modules over an exterior algebra. This was done for the case of a sheaf on $\PP^n$ in
\cite{EFS}. The case of products of projective spaces is treated in \cite{EES}. For a product of two or more projective
spaces there is major technical difficulty, in that the term in the free resolutions that the theory
associates to sheaves are generally all infinite-dimensional. In this package we compute only a sufficient
part of this infinite object.

The theory begins with two elementary observations:
\begin{itemize}
 \item If $W,M_1, M_2$ are finite-dimensional vector spaces over a field $k$, then there is a one-to-one correspondence between homomorphisms $W\otimes_kM_1\to M_2$ and homomorphisms
$ M_1 \to M_2 \otimes W^*$.
\item Consider the polynomial ring $S = \oplus_k \Sym_kW$ with generators in degree $1$ and the exterior algebra $E = \Lambda V= \oplus_k \Lambda^kV$, with $V=W^*$ and generators in degree -1.

If $M = \oplus_{i\geq i_0} M_i$ is a finitely generated graded module over $S$, then the sequence of maps 
$$
W\otimes M_i \to M_{i+1}
$$
 defining the module structure give rise, by the correspondence above,
to a sequence of maps $M_i\to M_{i+1}\otimes V$, which in turn correspond to a sequence of
linear maps of graded free $E$-modules 
$$ 
bgg(M): \cdots \to M_i\otimes \omega_E \to M_{i+1}\otimes \omega_E \to \cdots
$$
where $\omega_E:=\Lambda W=\Hom(E,K)$ is the free $E$-module generated in degree $n+1$, and as $M_j$ a $K$-vectorspace in degree $j$
\end{itemize}

One checks easily that the conditions of commutativity and associativity of the action of $S$ on $M$ exactly correspond to the condition that $bgg(M)$ is a complex; that is, consecutive maps compose to 0.

In exactly the same way, a graded $E$-module $N$ gives rise to a linear  complex of free $S$-modules
$$bgg(N): \ldots \to N_i \otimes S \to N_{i-1} \otimes  S \to \ldots $$


It is interesting to ask when these complexes are resolutions:

\begin{theorem}[Reciprocity] \cite[Theorem x.y] {EFS}. With notation as above:
\begin{enumerate}
 \item $bgg(M)$ is an injective resolution of the $E$-module $N$ if and only if
$bgg(N)$ is a free resolution of $M$.

\item These conditions are satisfies if and only if the Castelnuovo-Mumford regularity of $M$ is 0 and $M$ has no submodule of finite length.
\end{enumerate}
\end{theorem}


In their 1978 paper Bernstein-Gelfand-Gelfand \cite{BGG} used this correspondence  to identify  the derived category $D^b(\PP^n)$ of bounded complexes of coherent sheaves on $\PP^n$ with stable module category of $E$ module. In the same volume \cite{Bei} Beilinson  described
$D^b(\PP^n)$ in terms of exceptional sequences:
Both $\{ \sO(-k) | k=1,\ldots,n\}$ and 
$\{ \Lambda^k U | k=0, \ldots, n\}$ form  what is nowadays called  a semi-orthogonal full exceptional exceptional sequences for $D^b(\PP^n)$.
Here $U$ denotes the universal rank $n$ subbundle on $\PP^n=\PP(W)$:
$$ 0 \to U \to W\otimes \sO \to \sO(1) \to 0.$$ 
A key point is, that the diagonal $\Delta \subset \PP^n \times \PP^n$ has a Koszul resolution
$$ 
0 \to \Lambda^n U \boxtimes \sO(-n) \to \ldots \to U \boxtimes \sO(-1) \to \sO \to \sO_\Delta \to 0
$$
where $A \boxtimes B = pr_1^* A \otimes pr_2^* B$ for  coherent sheaf $A,B$ on $\PP^n$ denotes their tensor product on $\PP^n \times \PP^n$ after pull back along the first and second projection.

To connect these two approaches a key role is played by what we call the Tate resolution $\bf T(\sF)$: Let $M=\oplus M_d$ represent the coherent sheaf $\sF$ on $\PP^n$. Then the truncation $M_{\ge r} = \oplus_{d \ge r}M_d$ represents the same sheaf and the linear complex
$$bgg(M_{\ge r}) : M_r\otimes E(r) \to M_{r+1}\otimes E(r+1) \to \ldots$$ 
is acyclic if $r \gg 0$. Combining this injective resolution with a minimal free resolution of $P=\ker(M_r\otimes E(r) \to M_{r+1}\otimes E(r+1))$
yields an infinite exact complex
$$
\bT(\sF):   \ldots \to T^{r-2} \to T^{r-1} \to T^r \to T^{r+1} \to \ldots
$$
of free $E$-modules, which depends only on $\sF$. By \cite[Theorem x.y] {EFS} the terms of this complex are
$$\bT^d(\sF) =T^d= \sum_{i=0}^n H^i(\PP^n,\sF(d-i)) \otimes E(d-i).$$
 Hence syzygy computation over the exterior algebra $E$ allows us to compute all cohomology groups $H^i(\PP^n,\sF(k))$ in any bounded range $low \le k \le high$ for $low,high \in \ZZ$.
  The Beilinson monad $\bU(\sF)$ of $\sF$ is obtained from its Tate resolution by applying the functor
 
 
 $$\diagram
 \bU: &\{ \hbox{free $\omega_E$-modules}\} & \rTo & \{ \hbox{coherent sheaves on $\PP^n$} \}   \cr
  &\omega_E(i) & \mapsto & \Lambda^i U \subset \Lambda^i W \otimes \sO \cr
&\dTo^{a\cdot}&&   \dTo_{\neg a} \cr
&\omega_E(j) &\mapsto& \Lambda^j U \subset \Lambda^j W \otimes \sO \cr
\enddiagram
$$
 to $\bT(\sF)$ for $a \in \Lambda^{i-j} V \subset E$ a homogeneous element of degree $j-i$. Since this $\Lambda^i U \not=0$ for $0 \le i \le n$, the complex $\bU(\sF)$ depends only on a bounded part of $\bT(\sF)$.
 One of Beilinson's main theorems says that the complex $\bU(\sF)$ is exact except at position $0$:
 $$ \rH^* \bU(\sF) = \rH^0 \bU(\sF) \cong \sF.$$
 
 \medskip

The theory of Tate resolution on 
$$
\PP = \PP^{n_1}\times \ldots \times \PP^{n_t}
$$
 will similarly allow us to compute all the cohomology groups
$ H^j(\PP, \sF(a))$, for twists $a = (a_1,\ldots,a_t)$ in a bounded range
$$ low \le a \le high$$
for $low,high \in \ZZ^t$.
We consider again a pair of Koszul dual symmetric and exterior algebras:
If $\PP^{n_i}= \PP(W_i)$ then we consider the $\ZZ^t$-graded polynomial ring 
$$
S=\Sym(W_1 \oplus \ldots \oplus W_t)
$$
and the $\ZZ^t$-graded algebra $E= \Lambda(V_1 \oplus \ldots \oplus V_t)$ where $V_i =W_i^*$.

If $M$ is a graded $S$-module representing $\sF$ then the Tate resolution $\bT(\sF)$ is an
exact complex of free $E$-modules with terms
$$
\bT^d(\sF)= \sum_{a \in \ZZ^t} H^{d-|a|}(\PP,\sF(a)) \otimes \omega_E,
$$
where $|a|=a_1+\ldots+a_n$ denotes the total degree.
One of the difficulties to overcome is  that, when $t>1$, these $E$-modules are not finitely generated. We explain in Section \ref{TateRes}
how we can effectively compute the subquotient complex covering a finite  range
$$ 
low \le a \le high.
$$

There are some minor differences between the notation of the package and of \cite{EES}.  In the package:
\begin{itemize}
\item The exterior algebra E is positively graded.
\item We use E instead of $\omega_E.$
\item All complexes are chain complexes instead of cochain complexes.
\end{itemize}
Don't ask why!
%Challenge of toric varieties.

\section{Computing Cohomology Tables}

To compactly represent the dimensions of all the cohomology vector spaces of a given coherent sheaf $\sF$ we introduce the ``cohomology polynomial''
$$
\sum_{i\geq 0} (\dim H^i(\PP, \sF)h^i\in \ZZ[h].
$$
To shorten the notation, we usually write $H^i(\sF)$ in place of $H^i(\PP, \sF)$.
Let $M$ be a finitely generated $\ZZ^t$-graded $S$-module, and let $\sF$ be the 
corresponding sheaf on $\PP$. Let
$low\leq high\in \ZZ^t$ be multi-degrees defining an interval. We can compute
all the cohomology vector spaces of all the twists $\sF(a)$ for $low\leq a\leq high$ with the function 
\begin{verbatim}
 cohomologyHashTable(M,low,high)
\end{verbatim}
The output of this function is hash table consisting of the pairs
$$ 
a \Rightarrow \sum_{i\geq 0} (\dim H^i(\PP, \sF(a))h^i.
$$
In the case $t=2$ we can display the hash table as a matrix with function
\begin{verbatim}
 cohomologyMatrix(M,low,high)
\end{verbatim}
where the upper right hand corner of the table corresponds to the multi-index $high$.
For example, we can represent $\sF = \sO_{\PP^1\times \PP^2}$ with the module
$S^1$:
\begin{verbatim}
 (S,E) = productOfProjectiveSpaces{1,2}
 low  = {-3,-3};high = {3,3};
 cohomologyMatrix(S^1, low,high)
\end{verbatim}
which gives output
\begin{verbatim}
     | 20h 10h 0 10 20  30  40  |
     | 12h 6h  0 6  12  18  24  |
     | 6h  3h  0 3  6   9   12  |
     | 2h  h   0 1  2   3   4   |
     | 0   0   0 0  0   0   0   |
     | 0   0   0 0  0   0   0   |
     | 2h3 h3  0 h2 2h2 3h2 4h2 |
\end{verbatim}
As a further example we consider

     
\section{Computing the Beilinson Monad}

\begin{verbatim}
 
i20 :       T = tateResolution(M,low,high) 

       136      55      32      44      39      36      54      91      136      184      239      304      382      476      589
o20 = E    <-- E   <-- E   <-- E   <-- E   <-- E   <-- E   <-- E   <-- E    <-- E    <-- E    <-- E    <-- E    <-- E    <-- E
                                                                                                                              
      -8       -7      -6      -5      -4      -3      -2      -1      0        1        2        3        4        5        6

o20 : ChainComplex

i21 :       cohomologyMatrix(T,low,high)

o21 = | 28h  18h  8h  2  12  22  32  |
      | 20h  13h  6h  1  8   15  22  |
      | 12h  8h   4h  0  4   8   12  |
      | 4h   3h   2h  h  0   1   2   |
      | 4h2  2h2  0   2h 4h  6h  8h  |
      | 12h2 7h2  2h2 3h 8h  13h 18h |
      | 20h2 12h2 4h2 4h 12h 20h 28h |

                       7                7
o21 : Matrix (ZZ[h, k])  <--- (ZZ[h, k])

i22 :       B=beilinson T

       1      4
o22 = S  <-- S
              
      -1     0

o22 : ChainComplex

i23 :       M'=prune HH^0 B

o23 = cokernel {1, 1} | -x_(1,1) 0        0        -x_(0,1) |
               {0, 2} | x_(0,0)  x_(0,1)  0        0        |
               {1, 1} | -x_(1,0) 0        -x_(0,1) 0        |
               {1, 1} | 0        -x_(1,0) x_(0,0)  0        |
               {1, 1} | 0        -x_(1,1) 0        x_(0,0)  |
               {2, 0} | 0        0        -x_(1,1) x_(1,0)  |

                             6
o23 : S-module, quotient of S

i24 :       prune HH^1 B

o24 = cokernel | x_(1,1) x_(1,0) x_(0,1) x_(0,0) |

                             1
o24 : S-module, quotient of S

i25 :       isIsomorphic(M,M')

o25 = true
  \end{verbatim}


\section{Computing $R\pi_*$ of a Sheaf}
\begin{verbatim}
 
  Key
    directImageComplex
    (directImageComplex,Module,List)
    (directImageComplex,Ideal,Module,Matrix)
  Headline
    compute the direct image complex 
  Usage
    RpiM = directImageComplex(M,I)
    RphiN = directImageComplex(J,N,phi)
  Inputs
    M: Module
       representing a sheaf F on a product of projective spaces
    I: List
      corresponding to the factors to which pi projects
    J: Ideal
      the saturated ideal of a projective scheme X in some P^n
    N: Module
      representing a sheaf on X
    phi: Matrix
      a kx(m+1) matrix of homogeneous polynomials on P^n
      which define a morphism or rational map phi:X -> P^m,
      i.e. the 2x2 minors of phi vanish on X.      
  Outputs
    RpiM: ChainComplex
       a chain complex of modules over a symmetric algebra
    RphiN: ChainComplex
       a chain complex of modules over the coordinate ring of P^m
  Description
     Text
       Let M represent a coherent sheaf F on a product P=P^{n_0}x..xP^{n_{t-1}} 
       of t projective space. 
       
       Let $pi: P -> P^I= X_{i \in I} P^{n_i}$ denote the projection onto some factors. 
       We compute a chain complex of S_I modules whose
       sheafication is $Rpi_* F$. 
       
       The algorithm is based on the properties of strands,
       and the beilinson functor on $P^I$, see       
       @ HREF("http://arxiv.org/abs/1411.5724","Tate Resolutions on Products of Projective Spaces") @.
       Note that the resulting complex is a chain complex instead of a cochain complex,
       so that for example HH^1 RpiM is the module representing $R^1 pi_* F$
       
             In the second version we start with a projective scheme X =Proj(R/J) defined by J in some 
       P^n= Proj R with R \cong K[x_0..x_n] a polynomial ring,
       an  R-module N of representing a sheaf on X, and a matrix phi of homogeneous
       forms who's rows define a morphism phi: X -> P^m. In particular
       the 2x2 minors of phi vanish on X, and phi defines a morphism if and only if
       the entries of phi have no common zero in X.
       The algorithm passes to the graph of phi in P^n x P^m, and calls the first version
       of this function.


       As an example of the second version, we consider the ruled cubic surface scroll
       X subset P^4 defined by the 2x2 minors of the matrix
       $$ m= matrix \{ \{x_0,x_1,x_3\},\{x_1,x_2,x_4\} \},$$
       and the morphism f: X -> P^1 onto the base.
       f is defined by ratio of the two rows of m, hence by the 3x2 matrix phi=m^t.
       
       As a module N we take a symmetric power of the cokernel m, twisted by R^{d}.



i39 :        s=3,d=1

o39 = (3, 1)

o39 : Sequence

i40 :        N=symmetricPower(s,coker m)**R^{d};

i41 :        RphiN = directImageComplex(J,N,phi)

         ZZ              11        ZZ              9
o41 = (-----[x   , x   ])   <-- (-----[x   , x   ])
       32003  0,0   0,1          32003  0,0   0,1
                                 
      0                         1

o41 : ChainComplex

i42 :        T=ring RphiN

o42 = T

o42 : PolynomialRing

i43 :        netList apply(toList(min RphiN.. max RphiN),i-> 
      	   {-i, saturate annihilator HH^(-i) RphiN,betti res HH^(-i) RphiN})

      +--+--------+-----------+
      |  |        |        0 1|
o43 = |0 |ideal ()|total: 11 9|
      |  |        |    0: 11 9|
      +--+--------+-----------+
      |  |        |           |
      |-1|ideal 1 |total:     |
      +--+--------+-----------+

i44 :        R0=prune HH^0 RphiN

o44 = cokernel | -x_(0,1) 0        0        0        0        0        0        0        0        |
               | x_(0,0)  -x_(0,1) 0        0        0        0        0        0        0        |
               | 0        x_(0,0)  -x_(0,1) 0        0        0        0        0        0        |
               | 0        0        x_(0,0)  -x_(0,1) 0        0        0        0        0        |
               | 0        0        0        x_(0,0)  -x_(0,1) 0        0        0        0        |
               | 0        0        0        0        x_(0,0)  0        0        0        0        |
               | 0        0        0        0        0        -x_(0,1) 0        0        0        |
               | 0        0        0        0        0        x_(0,0)  -x_(0,1) 0        0        |
               | 0        0        0        0        0        0        x_(0,0)  -x_(0,1) 0        |
               | 0        0        0        0        0        0        0        x_(0,0)  -x_(0,1) |
               | 0        0        0        0        0        0        0        0        x_(0,0)  |

                             11
o44 : T-module, quotient of T

i45 :        dim R0, degree R0

o45 = (2, 2)

o45 : Sequence

i46 :        betti (sR0Dual = syz transpose presentation R0)

              0 1
o46 = total: 11 2
          0: 11 .
          1:  . .
          2:  . .
          3:  . 1
          4:  . 1

o46 : BettiTally

i47 :        saturate annihilator coker transpose sR0Dual

o47 = ideal 1

o47 : Ideal of T

i48 :        dual source sR0Dual       

       2
o48 = T

o48 : T-module, free, degrees {-4, -5}

i49 :        We conclude that the sheaf represented by R0 is O(-4)+O(-5) on P^1. 
\end{verbatim}



\section{The Tate Resolution and its properties}\label{TateRes}

The definition of the Tate resolution of a coherent sheaf $\sF$ on $\PP$ starts with a multigraded $S=\Sym(W_1 \oplus \ldots \oplus W_t)$-module $M$. At a first step we determine a muti-degree $c \in \ZZ^t$ such that $bgg(M_{ \ge c})$ is azyclic.
Then taking $N$ to be the kernel of the multiplication map
$$ M_c \otimes E \to \oplus_{j=1}^t M_{(c_1,\ldots,c_j+1,\ldots,c_t)} \otimes E $$
then the free resolution of $N(1,\ldots,1)$ is the part of the Tate resolution  with terms corresponding to cohomology groups
$H^p(\PP, \sF(a))$ for all $a \in \ZZ^t$ with $a_1\le c_1 +1, \ldots, a_t \le c_t+1$. 
Note that this is an infinite complex with finitely generated $E$ modules.
The complete Tate resolution is obtained by taking the limit of these complexes for $c \to \infty$.


strands etc

\bibliographystyle{ABC99}
\begin{thebibliography}{ABC99}

%\bibitem{Bass1963} Bass, Hyman
%On the ubiquity of Gorenstein rings.
%Math. Z. 82 1963 8\UTF{2013}28.



%\bibitem[A05]{A} M.~Aprodu: Remarks on syzygies of d-gonal curves Green's conjecture for curves,  Math. Res. Lett. 12 (2005), no. 2-3, 387--400.
%
%\bibitem[AF11]{AF} M.~Aprodu and G.~Farkas: Green's conjecture for curves on arbitrary {$K3$} surfaces, Compos. Math. 147 (2011), 839--851.
%
% \bibitem[AFPRW]{AFPRW} M.~Aprodu, G.~Farkas, S.~Papadima, C.~Raicu and J.~Weyman: In preparation.
%
%\bibitem[AB58]{AB} M.~Auslander and D.~Buchsbaum: Codimension and Multiplicity. Annals of Mathematics 68 (1958) 625--657.
%
%
%\bibitem[BE95]{BE} D.~Bayer and D.~Eisenbud: Ribbons and their canonical embeddings. Trans. Amer. Math. Soc. 347 (1995) 719--756. 
%
%
%\bibitem[BS15]{BS15}C.~Berkesch and F.-O.~Schreyer:
%Syzygies, finite length modules, and random curves, in:
%Commutative algebra and noncommutative algebraic geometry, {V}ol. {I},
%Math. Sci. Res. Inst. Publ.,
%67 (2015),
%25--52.
%
%\bibitem[B17]{Bopp} C.~Bopp: Canonical curves, Scrolls and K3 surfaces. \href{https://publikationen.sulb.uni-saarland.de/bitstream/20.500.11880/26917/1/phd_bopp.pdf}{Dissertation, Saarbr\"ucken Fall 2017}.
%
%\bibitem[BH15]{BH15} C.~Bopp and M.~Hoff: Resolutions of general canonical curves on rational normal scrolls. Archiv der Mathematik (Basel) 105 (2015) 239--249.
%
%\bibitem[BH17]{BH17} C.~Bopp and M.~Hoff: Moduli of lattice polarized K3 surfaces via relative canonical resolutions. Preprint \href{https://arxiv.org/abs/1704.02753}{arXiv:1704.02753}.
%
%
%
%
%\bibitem[BS18]{BS18} C.~Bopp and F.-O.~Schreyer:  A version of Green's conjecture over fields of finite characteristic. Preprint \href{https://arxiv.org/abs/1803.10481}{arXiv:1803.10481}.
%
%\bibitem[BH93]{BH93} W.~Bruns and J.~Herzog, J\"urgen:
%Cohen-{M}acaulay rings
%Cambridge Studies in Advanced Mathematics 39, Cambridge University Press. xi, 403 p. (1993). 
%
%\bibitem[D15]{D} A.~Deopurkar: The canonical syzygy conjecture for ribbons. \href{https://arxiv.org/abs/1510.07755}{arXiv:1510.07755}
%
%\bibitem[DFS16]{DFS} A.~Deopurkar, M.~ Fedorchuk and D.~Swinarski: Toward GIT stability of syzygies of canonical curves.
% Algebr. Geom.  3  (2016),  no. 1, 1--22.
%
%\bibitem[E97]{E} D.~Eisenbud: Commutative Algebra with a View toward Algebraic Geometry. GTM 150, Springer-Verlag NY, 1997.
%
%\bibitem[E05]{E05} D.~Eisenbud: The Geometry of Syzygies. GTM 229, Springer-Verlag NY, 2005.
%
%\bibitem[EH87]{EH} D.~Eisenbud and J.~Harris: Varieties of Minimal Degree (a centennial account). Algebraic geometry, 
%Proc. Sympos. Pure Math., 46, Part 1, pp. 3-13. Amer. Math. Soc., Providence, RI, 1987. 
%
%
%
%\bibitem[EiSa]{EiSa} D.~Eisenbud and A.~Sammartano: Correspondence Scrolls. In preparation.
%
%\bibitem[ES18]{ES18} D.~Eisenbud and F.-O.~Schreyer: K3Carpets , a Macaulay2 package to investigate K3 carpets.
%Available at\\ \href{https://www.math.uni-sb.de/ag/schreyer/index.php/computeralgebra}{https://www.math.uni-sb.de/ag/schreyer/index.php/computeralgebra}.
%
%\bibitem[EMSS16]{EMSS} B.~Erocal, O.~Motsak, F.-O.~Schreyer and A.~Steenpass: 
%Refined Algorithms to Compute Syzygies. J. Symb. Comput. 74 (2016), 308--327.
%
%\bibitem[F69]{F69} W.~Fulton: Hurwitz schemes and irreducibility of moduli of algebraic curves. Ann. Math. (2)
%90 (1969), 542--575.
%
%\bibitem[GP97]{GP}  F.J.~Gallego and B.~Purnaprajna, Degenerations of K3 surfaces in projective space.
%Trans. Amer. Math. Soc. 349 (1997) 2477--2492.
%
%\bibitem[M2]{M2} D.R.~ Grayson and M.E.~ Stillman,
%          Macaulay2, a software system for research in algebraic geometry.
%          Available at \href{https://faculty.math.illinois.edu/Macaulay2/}{https://faculty.math.illinois.edu/Macaulay2/}
%      
%\bibitem[M95]{Muk}   S.~Mukai, Curves and symmetric spaces I. Amer. J. Math 117 (1995), 1627--1644.
%
%\bibitem[S86]{S86} F.-O.~Schreyer: Syzygies of canonical curves and special linear series. Math. Ann. 275 (1986), 105--137.
%
%\bibitem[S88]{S88} F.-O.~Schreyer: Green's conjecture for the general p-gonal curve of large genus. In: Algebraic curves and projective geometry, Proceedings Trento 1988, Springer Lecture Notes 1389, 254--260.
%
%\bibitem[S91]{S91} F.-O.~Schreyer: A standard basis approach to syzygies of canonical curves. J. reine angew. Math. 421 (1991), 83--123.
%
%
%\bibitem[V05]{V05} C.~Voisin: 
%Green's canonical syzygy conjecture for generic curves of odd genus. Compos. Math. 141 (2005) 1163--1190. 

\end{thebibliography}

%\bigskip
%
%%\author[David Eisenbud]{David Eisenbud}
%%\address{ Department of mathematics, University of California at Berkeley, Berkeley, CA 94720, USA, and 
%%The Mathematical Sciences Research Institute}
%%\email{de@msri.org}
%%
%%\author[Frank-Olaf Schreyer]{Frank-Olaf Schreyer}
%%\address{Fachbereich Mathematik, Universit\"at des Saarlandes, Campus E2 4, D-66123 Saar\-br\"ucken, Germany}
%%\email{schreyer@math.uni-sb.de}
%%
%%\vbox{\noindent Author Addresses:\par
%%\smallskip
%%\noindent{David Eisenbud}\par
%%\noindent{Mathematical Sciences Research Institute,
%%Berkeley, CA 94720, USA}\par
%%\noindent{de@msri.org}\par
%%}

\end{document}



